\subsection{Data Storage \& Management}

\subsubsection{Postgres}

\begin{itemize}
    \item \textbf{Đặc điểm}:
    \begin{itemize}
        \item Dữ liệu được lưu trữ trong các bảng với cấu trúc (schema) cố định.
        \item Ràng buộc kiểu dữ liệu, khóa chính, khóa ngoại, ... giúp đảm bảo tính nhất quán.
    \end{itemize}
    \item \textbf{Ví dụ Code (SQL)}: 
        \begin{lstlisting}[language=sql]
CREATE TABLE users (
    id SERIAL PRIMARY KEY,
    name VARCHAR(100) NOT NULL,
    email VARCHAR(100) UNIQUE NOT NULL
);\end{lstlisting}
\end{itemize}

\noindent
\subsubsection{MongoDB}

\begin{itemize}
    \item \textbf{Đặc điểm}:
    \begin{itemize}
        \item Dữ liệu được lưu dưới dạng document (JSON/BSON) trong các collection.
        \item Schema linh hoạt, dễ dàng mở rộng với dữ liệu phi cấu trúc.
    \end{itemize}
    \item \textbf{Ví dụ Code (Mongo Shell)}: 
        \begin{lstlisting}[language=java]
db.users.insertOne({
    name: "John Doe",
    email: "john.doe@example.com"
});\end{lstlisting}
\end{itemize}

\subsubsection{So sánh}
\begin{itemize}
    \item \textbf{Schema cố định so sánh linh hoạt}: \textbf{Postgres} yêu cầu định nghĩa cấu trúc dữ liệu trước, còn \textbf{MongoDB} cho phép lưu trữ dữ liệu không đồng nhất.
    \item \textbf{Quản lý dữ liệu}: Các ràng buộc (constraints) mạnh mẽ của \textbf{Postgres} hỗ trợ tính toàn vẹn dữ liệu, trong khi \textbf{MongoDB} cung cấp tính mở rộng dễ dàng.
\end{itemize}
